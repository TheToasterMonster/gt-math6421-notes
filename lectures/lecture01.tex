\chapter{Aug.~19 --- Affine Varieties}

\section{Motivation for Algebraic Geometry}

\begin{remark}
  Why study algebraic geometry? Algebraic
  geometry connects to many fields of math.
\end{remark}

\begin{example}
  Consider a plane curve
  $\{f(z_1, z_2) = 0\} \subseteq \C^2$, e.g.
  an elliptic curve
  $z_2^2 - z_1^3 + z_1 - 1 = 0$.
  Compactify and set $C$ to be
  the closure of $C^0$ in $\C \PP^2$, and
  let $d = \deg f$. There are connections in
  \begin{enumerate}
    \item Topology: $H^1(C, \C) \cong \C^{2g}$,
      where $g = (d - 1)(d - 2) / 12$;
    \item Arithmetic: the number of
      $\Q$-points is finite if $d > 3$;
    \item Complex geometry:
      We have $C \cong \C \PP^2$ for
      $d = 1, 2$, $C \cong \C / \Lambda$
      for $d = 3$,
      and $C \cong \mathbb{H} / \Gamma$
      for $d > 3$.
  \end{enumerate}
\end{example}

\section{Affine Varieties}

Fix an algebraically closed field $k$ (e.g.
$\C$, $\overline{\Q}$, $\overline{\F}_p$,
etc.).

\begin{definition}
  \emph{Affine space} is the set
  $\Affine^n = \Affine^n_k = \{\vec{a} = (a_1, \dots, a_n) : a_i \in k\}$.
\end{definition}

\begin{remark}
  Note the following:
  \begin{enumerate}
    \item $\Affine^n_k$ is the same set
      as $k^n$, but forgetting the vector
      space structure;
    \item $f \in k[x_1, \dots, x_n]$
      gives a polynomial function
      $\Affine^n_k \to k$ by evaluation:
      $a \mapsto f(a)$.
  \end{enumerate}
\end{remark}

\begin{definition}
  For a subset $S \subseteq k[x_1, \dots, x_n]$,
  its \emph{vanishing set} is
  \[
    V(S) = \{
      a \in \Affine^n : f(a) = 0 \text{ for all } f \in S
    \}.
  \]
  An \emph{affine variety} is a subset of
  $\Affine^n_k$ of this form.
\end{definition}

\begin{example}
  Consider the following:
  \begin{enumerate}
    \item $\Affine^n = V(\varnothing) = V(\{0\})$;
    \item $\varnothing = V(1) = V(k[x_1, \dots, x_n])$;
    \item a point $a = (a_1, \dots, a_n)$
      is an affine variety:
      $V(\{x_1 - a_1, \dots, x_n - a_n\}) = \{a\}$;
    \item a linear space
      $L \subseteq \Affine^n$ (it is the
      kernel of some matrix);
    \item plane curves $V(f(x, y)) \subseteq \Affine^2_{x, y}$;
    \item $\mathrm{SL}_n(k) \subseteq \Affine^{n \times n}$ is an
      affine variety: $\mathrm{SL}_n(k) = V(\det([x_{i, j}]) - 1)$;
    \item $\mathrm{GL}_n(k)$ (as a set) is an
      affine variety in $\Affine^{n \times n + 1}$:
      $\GL_n(k) = V(\det([x_{i, j}])y - 1)$;
    \item if $X \subseteq \Affine^m$ and
      $Y \subseteq \Affine^n$ are affine
      varieties, then $X \times Y \subseteq \Affine^{m + n}$
      is an affine variety;
    \item the affine varieties
      $X \subseteq \Affine^1_k$ are of
      the form: finite set of points,
      $\varnothing$, or $\Affine^1_k$.
  \end{enumerate}
\end{example}

\begin{prop}[Relation to ideals]
  If $S \subseteq k[x_1, \dots, x_n]$, then
  $V(S) = V(\langle S \rangle)$, where
  $\langle S \rangle$ is the ideal generated
  by $S$.
\end{prop}

\begin{proof}
  Since $S \subseteq \langle S \rangle$,
  we have $V(\langle S \rangle) \subseteq V(S)$.
  Conversely, if $f, g \in S$ and
  $h \in k[x_1, \dots, x_n]$, then
  $f + g$ and $hf$ vanish on $V(S)$, so
  we see that
  $V(S) \subseteq V(\langle S \rangle)$.
\end{proof}

\begin{remark}
  The statement implies that if
  $f_1, \dots, f_r \in k[x_1, \dots, x_n]$,
  then $V(f_1, \dots, f_r) = V((f_1, \dots, f_n))$.
  The following are some further
  applications of the statement:
  \begin{enumerate}
    \item affine varities are vanishing
      loci of ideals;
    \item if $X \subseteq \Affine^n$ is an
      affine variety, then $X$ is cut out
      by finitely many polynomial
      equations.
  \end{enumerate}
  To see the second statement, note that
  $X = V(I)$ for some ideal $I \le k[x_1, \dots, x_n]$.
  By the Hilbert basis theorem that
  $k[x_1, \dots, x_n]$ is Noetherian,
  there are finitely many
  $f_1, \dots, f_r \in k[x_1, \dots, x_n]$
  such that $I = (f_1, \dots, f_r)$.
  So $X = V(I) = V(f_1, \dots, f_r)$.
\end{remark}

\begin{prop}[Properties of the vanishing set]
  For ideals $I, J$ of $k[x_1, \dots, x_n]$,
  \begin{enumerate}
    \item if $I \subseteq J$, then
      $V(J) \subseteq V(I)$;
    \item $V(I) \cap V(J) = V(I + J)$;
    \item $V(I) \cup V(J) = V(IJ) = V(I \cap J)$.
  \end{enumerate}
\end{prop}

\begin{proof}
  $(1)$ This follows from definitions
  and actually holds for general subsets.

  $(2)$ Note that
  $V(I) \cap V(J) = V(I \cap J) = V(\langle I \cup J \rangle) = V(I + J)$.

  $(3)$
  We only prove the first equality, the
  second is similar.
  Recall that $IJ = \big\{{\sum_{i = 1}^r f_i g_i : f_i \in I, g_i \in J}\big\}$.
  We have the forwards inclusion
  $V(I) \cup V(J) \subseteq V(IJ)$ from
  definitions. For the reverse inclusion,
  consider a point $x \notin V(I) \cup V(J)$.
  So there exists $f \in I$ and
  $g \in J$ such that $f(x), g(x) \ne 0$.
  So $f(x)g(x) \ne 0$, which implies
  that $x \notin V(IJ)$. Thus
  $V(IJ) \subseteq V(I) \cup V(J)$ as well.
\end{proof}

\begin{remark}
  The above implies that if $X$ and $Y$
  are affine varieties in $\Affine^n_k$,
  then so are $X \cup Y$ and $X \cap Y$.
\end{remark}

\begin{example}
  Consider $V(y^2 - x^2, y^2 + x^2) \subseteq \Affine^2$.
  Note that $(y^2 - x^2, y^2 + x^2) = (x^2, y^2)$,
  from which we can easily see that
  $V(y^2 - x^2, y^2 + x^2) = V(x^2, y^2) = \{0\}$.
\end{example}

\section{Correspondence with Ideals}

\begin{remark}
  Our goal is to build a correspondence
  between affine varieties in $\Affine^n_k$
  and ideals of $k[x_1, \dots, x_n]$.
\end{remark}

\begin{definition}
  For a subset $X \subseteq \Affine^n_k$,
  define
  \[
    I(X) = \{f \in k[x_1, \dots, x_n] : f(a) = 0
      \text{ for all } a \in X\}.
  \]
\end{definition}

\begin{remark}
  Note that $I(X)$ is in fact an ideal of
  $k[x_1, \dots, x_n]$.
\end{remark}

\begin{example}
  Consider the following:
  \begin{enumerate}
    \item $I(\varnothing) = k[x_1, \dots, x_n]$;
    \item $I(\Affine^n_k) = \{0\}$, this
      will follow from the
      Hilbert nullstellensatz and
      relies on $k = \overline{k}$ (for
      $k = \R$, the polynomial $x^2 + y^2 + 1$ is
      always nonzero and thus lies in
      $I(\Affine^n_\R)$);
    \item for $n = 1$, if
      $S \subseteq \Affine^1_k$ be an
      infinite set, then $I(S) = (0)$.
    \item for $n = 1$, we have
      $I(V(x^2)) = I(\{0\}) = (x)$.
  \end{enumerate}
\end{example}

\begin{remark}
  What properties does $I(X)$ satisfy?
\end{remark}

\begin{definition}
  Let $R$ be a ring.
  The \emph{radical} of an ideal
  $J \le R$ is
  \[
    \sqrt{J} = \{
      f \in R : f^n \in J \text{ for some }
      n > 0
    \}.
  \]
  An ideal $J$ is \emph{radical} if
  $J = \sqrt{J}$.
\end{definition}

\begin{exercise}
  Check the following:
  \begin{enumerate}
    \item $\sqrt{J}$ is always an ideal.
    \item $\sqrt{\sqrt{J}} = \sqrt{J}$.
    \item An ideal $J \le R$ is radical
      if and only if $R / J$ is
      reduced.\footnote{Recall that a ring $R$ is \emph{reduced} if for all nonzero $f \in R$ and positive integers $n$, we have $f^n \ne 0$. It is immediate that an integral domain is reduced.}
  \end{enumerate}
\end{exercise}

\begin{prop}
  If $X \subseteq \Affine^n_k$ is a subset
  (not necessarily an affine variety),
  then $I(X)$ is radical.
\end{prop}

\begin{proof}
  Fix $f \in k[x_1, \dots, x_n]$. If
  $f^n \in I(X)$, then $f^n(x) = 0$ for
  all $x \in X$. This implies
  $f(x) = 0$ for all $x \in X$, so
  $f \in I(X)$. Thus we see that
  $I(X) = \sqrt{I(X)}$.
\end{proof}

\begin{theorem}[Hilbert's nullstellensatz]
  \label{thm:hilbert-nullstellensatz}
  If $J \le k[x_1, \dots, x_n]$ is an
  ideal, then $I(V(J)) = \sqrt{J}$.
\end{theorem}

\begin{example}
  Let $n = 1$, so that $k[x]$ is a PID. Let $f = (x - a_1)^{m_1} \cdots (x - a_r)^{m_r}$.
  Then
  \[
    I(V(f)) = I(\{a_1, \dots, a_r\})
    = ((x - a_1)^{m_1} \cdots (x - a_r)^{m_r}).
  \]
\end{example}
