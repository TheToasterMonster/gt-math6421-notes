\chapter{Sept.~4 --- Regular Functions}

\section{Properties of Regular Functions}

\begin{prop}\label{prop:regular-function-properties}
  Let $X$ be an affine variety and
  $U \subseteq X$ open. Then:
  \begin{enumerate}
    \item if $\varphi \in \OO_X(U)$,
      then $V(\varphi) = \{x \in U : \varphi(x) = 0\}$
      is closed in $U$;
    \item (identity principle) If
      $X$ is irreducible, $U \subseteq X$
      is nonempty and open, and
      $\varphi, \psi \in \OO_X(U)$ with
      $\varphi|_W = \psi|_W$ for some
      $W \subseteq U$ nonempty and open,
      then $\varphi = \psi$ in $\OO_X(U)$.
  \end{enumerate}
\end{prop}

\begin{proof}
  $(1)$ It suffices to show that
  $U \setminus V(\varphi)$ is open in $U$.
  Fix $a \in U \setminus V(\varphi)$.
  Since $\varphi$ is regular, there exists
  an open neighborhood $a \in U_a \subseteq U$
  and $f_a, g_a \in A(X)$ such that
  \[
    \varphi|_{U_a} = \frac{g_a}{f_a}.
  \]
  So $a \in \{g_a \ne 0\} \cap U_a \subseteq U \setminus V(\varphi)$. This is
  an open set containing $a$ in
  $U \setminus V(\varphi)$, so
  $U \setminus V(\varphi)$ is open.

  $(2)$ Since $X$ is irreducible, $U$
  is also irreducible. The locus
  $\{x \in U : \varphi(x) = \psi(x)\} = V(\varphi - \psi)$
  is closed in $U$ by $(1)$. It also
  contains $W$. Since $W$ is dense
  (it is a nonempty open set in an irreducible
  topological space), we must have
  $V(\varphi - \psi) = U$.
  This proves the claim.
\end{proof}

\begin{example}
  In $(2)$ of Proposition \ref{prop:regular-function-properties}, the
  assumption that $X$ is irreducible is
  necessary. Consider
  \[
    U = X = V(xy) \subseteq \Affine_k^2
    \quad \text{and} \quad
    W = V(xy) \setminus V(x).
  \]
  Then the regular functions
  $\varphi = x$ and $\psi = x + y$
  agree on $W$ but are not equal on $U$.
\end{example}

\section{Distinguished Open Sets}
\begin{remark}
  We will see that an affine variety
  has a basis of open sets on which
  we can compute $\OO_X(U)$.
\end{remark}

\begin{definition}
  A \emph{distinguished open set} of
  an affine variety $X$ is a subset of
  the form
  \[
    D(f) = X \setminus V(f)
  \]
  for some polynomial function $f \in A(X)$.
\end{definition}

\begin{remark}
  We have the following:
  \begin{enumerate}
    \item The $D(f)$ are closed under
      (finite) intersection: $D(fg) = D(f) \cap D(g)$.
    \item The $D(f)$ form a basis for the
      Zariski topology on $X$: If $U \subseteq X$ is
      open, then $U = X \setminus V(f_1, \dots, f_r)$
      for some $f_1, \dots, f_r \in A(X)$
      (since $X$ is Noetherian). So
      $U = D(f_1) \cup \cdots \cup D(f_r)$.
  \end{enumerate}
\end{remark}

\begin{remark}
  We will view $D(f)$ as
  ``small open sets'' (under mild
  assumptions, $\codim_X (X \setminus D(f)) = 1$).
\end{remark}

\begin{theorem}\label{thm:regular-functions-on-distinguished-open-sets}
  If $X$ is an affine variety and
  $f \in A(X)$, then
  \[
    \OO_X(D(f))
    = \left\{
      \frac{g}{f^m} : g \in A(X), m \ge 0
    \right\}.
  \]
\end{theorem}

\begin{proof}
  We have an injective ring homomorphism
  \[
    \left\{
      \frac{g}{f^m} : g \in A(X), m \ge 0
    \right\}
    \longrightarrow \OO_X(D(f)),
  \]
  it suffices to show this map is
  surjective. Fix $\varphi \in \OO_X(D(f))$.
  For any $a \in D(f)$, there exists an
  open neighborhood $a \in U_a \subseteq D(f)$
  and $f_a, g_a \in A(X)$ such that
  $\varphi|_{U_a} = g_a / f_a$.
  We may further assume that
  \begin{enumerate}
    \item $U_a = D(h_a)$ for
      some $h_a \in A(X)$ (by shrinking
      $U_a$ if necessary, since the
      $D(h)$ form a basis);
    \item $h_a = f_a$ (by rewriting
      $g_a / f_a = g_a h_a / f_a h_a$ and
      replacing $h_a, f_a$ with $f_a h_a$).
  \end{enumerate}
  Then for $a, b \in D(f)$, we have
  $f_a g_b = f_b g_a$ on $D(f_a) \cap D(f_b)$.
  Since both the left and right hand
  sides
  vanish on $X \setminus (D(f_a) \cap D(f_b))$,
  we have $f_a g_b = f_b g_a$ in $A(X)$.
  Now we can write
  \[
    V(f) = \bigcap_{a \in D(f)} V(f_a)
    = V(f_a : a \in D(f)),
  \]
  so $f \in I(V(f_a : a \in D(f)))$.
  By the Nullstellensatz, there exists
  $n \ge 0$ such that
  \[
    f^n = \sum_{a \in D(f)} k_a f_a, \quad
    k_a \in A(X),
  \]
  where only finitely many of the $k_a$
  are nonzero. Set
  $g = \sum_{a \in D(f)} k_a g_a$, and
  we claim that $\varphi = g / f^n$. To
  see this, note that on $U_b$, we have
  $\varphi|_{U_b} = g_b / f_b$. Now
  since $f_a g_b = f_b g_a$, we have
  \[
    g f_b
    = \sum_{a \in D(f)} k_a g_a f_b
    = \sum_{a \in D(f)} k_a f_a g_b
    = f^n g_b,
  \]
  which shows that $\varphi|_{U_b} = (g / f^n)|_{U_b}$.
  Since this holds for any $U_b$,
  we have $\varphi = g / f^n$ in
  $\OO_X(D(f))$.
\end{proof}

\begin{remark}
  Theorem \ref{thm:regular-functions-on-distinguished-open-sets}
  has the following consequences:
  \begin{enumerate}
    \item The $f = 1$ case implies that the
    natural ring homomorphism
    $A(X) \to \OO_X(X)$ is surjective
    and hence an isomorphism (note that
    $D(1) = X$).
    \item We will see that
      $\OO_X(D(f)) \cong A(X)_f$,
      the \emph{localization} of $A(X)$ at $f$.
  \end{enumerate}
\end{remark}

\begin{example}
  How do we compute $\OO_X(U)$ on
  non-distinguished open sets? Consider
  \[
    X = \Affine_k^2 \quad \text{and} \quad
    U = \Affine_k^2 \setminus \{(0,0)\}.
  \]
  Note that $U$ is never a distinguished
  open set. We claim that the ring
  homomorphism
  \[
    k[x, y] \longrightarrow \OO_{\Affine_k^2}(\Affine_k^2 \setminus \{(0,0)\})
  \]
  is an isomorphism. The map is
  injective by the identity principle,
  so it suffices to show surjectivity.
  The strategy is use
  $U = D(x) \cup D(y)$ (in general, cover
  $U$ by basis elements). Fix
  $\varphi : U \to k$ regular, so
  \begin{align*}
    \varphi|_{D(x)}
    = \frac{f}{x^m} \quad &\text{for some } f \in k[x, y], m \ge 0 \\
    \varphi|_{D(y)}
    = \frac{g}{y^n} \quad & \text{for some } g \in k[x, y], n \ge 0.
  \end{align*}
  Since we are in a UFD, we may assume
  that
  $x \nmid f$ and $y \nmid g$.
  Now $fy^n = gx^m$ on $D(y) \cap D(x)$,
  so by the identity principle,
  $fy^n = gx^m$ on $\Affine_k^2$, so
  $fy^n = gx^m$ in $k[x, y]$.
  Using that $y \nmid g$, $x \nmid f$,
  and that $k[x, y]$ is a UFD, we must
  have $n = m = 0$, hence $f = g$. In
  particular, we have
  \[
    \varphi|_{D(x)} = \varphi|_{D(y)} = f,
  \]
  so the map $k[x, y] \to \OO_{X}(U)$
  is surjective.
\end{example}

\section{Localization}

\begin{remark}
  We want to invert a subset of a ring,
  in particular \emph{multiplicative systems}.
\end{remark}

\begin{definition}
  A \emph{multiplicative system}
  of a ring $A$ is a subset such that
  \begin{enumerate}
    \item $1 \in S$;
    \item $S$ is closed under
      multiplication.
  \end{enumerate}
\end{definition}

\begin{example}
  The following examples of $S$ are
  multiplicative systems:
  \begin{enumerate}
    \item $S = A$ or $S = \{1\}$;
    \item if $\p \le A$ is a prime ideal,
      then $S = A \setminus \p$;
    \item if $f \in A$, then
      $S = \{f^m : m \ge 0\}$.
  \end{enumerate}
\end{example}

\begin{definition}
  The \emph{localization} of a ring
  $A$ at a multiplicative system $S$ is
  the ring
  \[
    S^{-1} A
    = \left\{\frac{a}{s} : a \in A, s \in S\right\}/\sim
  \]
  where the $a / s$ are formal symbols
  with $a / s \sim a' / s'$ if
  $t(as' - a' s) = 0$ for some
  $t \in S$.\footnote{Note that if $A$ is a domain and $0 \notin S$, then this condition is equivalent to $as' = a's$.}
  The operations are given by
  the usual addition and multiplication
  of fractions:
  \[
    \frac{a}{s} \cdot \frac{a'}{s'}
    = \frac{a a'}{s s'} \quad \text{and} \quad
    \frac{a}{s} + \frac{a'}{s'}
    = \frac{a s' + a' s}{s s'}.
  \]
  Check as an exercise that these
  operations respect the equivalence
  relation.
\end{definition}

\begin{example}
  The following are examples of localization:
  \begin{enumerate}
    \item If $A$ is a domain and
      $S = A \setminus \{0\}$, then
      $S^{-1} A = \Frac A$.
    \item If $S = \langle f \rangle = \{1, f, f^2, \dots\}$, then
      we will write
      $A_f = S^{-1} A$.
    \item If $S = A \setminus \p$ for
      a prime ideal $\p$, then
      we will write $A_\p = S^{-1} A$.
  \end{enumerate}
\end{example}

\begin{prop}
  We have the following properties
  of localization:
  \begin{enumerate}
    \item (Universal property of localization)
      For any ring homomorphism
      $\varphi : A \to B$ such that
      $\varphi(s)$ for all $s \in S$,
      then there exists a unique
      ring homomorphism which makes
      the following diagram commute:
      \begin{center}
      \begin{tikzcd}
        A \ar[rr, "\varphi"] \ar[rd, swap, "\pi : a \mapsto a / 1"] & & B \\
        & S^{-1} A \ar[ur, dashed, swap, "\exists !"]
      \end{tikzcd}
      \end{center}
    \item There is a bijection
      between the prime ideals
      $\p \le A$ with $\p \cap S = \varnothing$
      and the prime ideals $\q \le S^{-1} A$
      given by $\p \mapsto \pi(\p) S^{-1} A$
      with inverse $\q \mapsto \pi^{-1}(\q)$,
      where $\pi : A \to S^{-1} A$ is
      the map $a \mapsto a / 1$.
  \end{enumerate}
\end{prop}

\begin{remark}
  In more generality, for an $A$-module $M$, we
  can define the localization $S^{-1} M$,
  which is an $S^{-1} A$-module. This gives
  a functor $\mathrm{Mod}_A \to \mathrm{Mod}_{S^{-1}A}$
  which is exact.
\end{remark}
