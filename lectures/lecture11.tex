\chapter{Sept.~23 --- Pre-varieties, Part 2}

\section{More on Pre-varieties}

\begin{prop}
  Let $(X, \OO_X)$ be a pre-variety.
  \begin{enumerate}
    \item $X$ is Noetherian as a
      topological space.
    \item $X$ has a basis by affine varieties.
  \end{enumerate}
\end{prop}

\begin{proof}
  (1) Note that
  $X$ has a finite cover by
  affine varieties $U_i$, which are
  each Noetherian.

  (2) If $(X, \OO_X)$ is affine, then
  $\{D(f) : f \in \OO_X(X)\}$ gives such
  a basis. Do this for each
  $U_i$.
\end{proof}

\begin{example}[General gluing procedure]
  Let $I$ be a finite index set,
  $(X_i, \OO_{X_i})$ affine varieties,
  $U_{i, j} \subseteq X_i$ open sets, and
  $f_{i, j} : U_{i, j} \to U_{j, i}$
  isomorphisms
  for each $i, j \in I$, satisfying
  \begin{enumerate}
    \item $U_{i, i} = X_i$ and
      $f_{i, i} = \id$;
    \item $f_{i, j}^{-1}(U_{j, i} \cap U_{j, k}) = U_{i, j} \cap U_{i, k}$;
    \item the following diagram commutes:
      \begin{center}
        \begin{tikzcd}
          U_{i, j} \cap U_{i, k} \ar[dr, "f_{i, j}"] \ar[rr, "f_{i, k}"]
          & & U_{k, i} \cap U_{k, j} \\
          & U_{j, i} \cap U_{j, k} \ar[ur, "f_{j, k}"]
        \end{tikzcd}
      \end{center}
  \end{enumerate}
  We can define
  $X = (\bigsqcup_{i \in I} X_i) / {\sim}$
  with the quotient topology, where
  $a \sim a$ for $a \in X_i$ and
  $a \sim f_{i, j}(a)$ for $a \in U_{i, j}$.
  The inclusions $j_i : X_i \hookrightarrow X$
  are open embeddings, and we can set
  \[
    \OO_X(U)
    := \{
      \varphi : U \to k \mid
      \varphi|_{X_i} = j_i^* \varphi
      \text{ is regular for all } i
    \}.
  \]
  Then $(X, \OO_X)$ is a ringed space
  with $(j_i(X_i), \OO_X|_{j_i(X_i)}) \cong (X_i, \OO_{X_i})$,
  so $(X, \OO_X)$ is a pre-variety.
\end{example}

\begin{remark}
  Any pre-variety $X$ is a gluing of
  affine varieties.
  To see this, note that there exists a
  cover $X = \bigcup_{i = 1}^s U_i$
  by affine varieties. Then we can
  take $X_i = U_i$,
  $U_{i, j} = U_i \cap U_j \subseteq X_i$,
  and $f_{i, j} : U_{i, j} \to U_{j, i}$
  to be the identity map.
\end{remark}

\begin{prop}
  Let $X$ be a pre-variety.
  \begin{enumerate}
    \item If $U \subseteq X$ is an open
      set, then $(U, \OO_X|_U)$ is again
      a pre-variety.
    \item Let $Z \subseteq X$ be a closed
      set. For $U \subseteq Z$ open, set
      \[
        \OO_Z(U) = \left\{\varphi : U \to k 
          \mid
          \substack{\displaystyle\text{for each $a \in U$, there
          exists open $a \in W \subseteq X$
        and\,} \\ \displaystyle \text{$\psi : W \to k$ regular
      such that $\varphi|_{W \cap Z} = \psi|_{W \cap Z}$}
  }\right\}.
      \]
      Then $(Z, \OO_Z)$ is a pre-variety.
  \end{enumerate}
\end{prop}

\begin{proof}
  (1) Note that $X$ has a basis by affine
  varieties, so we can cover $U$ by
  affine varieties. This cover may be
  infinite, but we can pass to a
  finite subcover since $X$ and hence
  $U$ is Noetherian.

  (2) The idea is to first reduce to the case
  $X = V(I) \subseteq \Affine^n$, so
  $Z \subseteq X$ is cut out by polynomials.
  Then observe that $\OO_Z$ agrees with
  the previous definition.
\end{proof}

\begin{remark}
  Note that unions of intersections
  of closed and open sets are not
  necessarily pre-varieties.
  For instance, consider
  $(\Affine^2 \setminus V(xy)) \cup \{0\}$.
\end{remark}

\begin{prop}
  If $X, Y$ are pre-varieties, then
  there exists a pre-variety with morphisms
  \begin{center}
    \begin{tikzcd}
      & P \ar[dl, "p_1", swap] \ar[dr, "p_2"] \\
      X & & Y
    \end{tikzcd}
  \end{center}
  with the property that for every diagram
  \begin{center}
    \begin{tikzcd}
      & Z \ar[ddl, "f_X", bend right=30, swap] \ar[ddr, "f_Y", bend left=30] \ar[d, "f", dashed, swap] \\
      & P \ar[dl, "p_1"] \ar[dr, "p_2", swap] \\
      X & & Y
    \end{tikzcd}
  \end{center}
  there exists a unique morphism
  $f$ such that the diagram commutes.
  We call $P$ the \emph{product} of
  $X$ and $Y$, and write
  $X \times Y := P$. Moreover,
  set theoretically
  $X \times Y = \{(x, y) : x \in X \text{ and } y \in Y\}$.
\end{prop}

\begin{proof}
  We know the result holds when
  $X, Y, Z$ are affine or even
  open sets of affine varieties. In the
  general case, fix an open affine cover
  $X = \bigcup_{i = 1}^s U_i$
  and $Y = \bigcup_{j = 1}^r V_j$. Then
  glue the products by:
  \begin{enumerate}
    \item $P_{(i, j)} := U_i \times V_j$,
    \item along $P_{(i, j), (i', j')} :
      (U_i \cap U_{i'}) \times (V_j \cap V_{j'})$,
    \item via
      $f_{(i, j), (i', j')} :
      P_{(i, j), (i', j')} \overset{\cong}{\to}
      P_{(i', j'), (i, j)}$, the
      isomorphism from the universal
      property of products.
  \end{enumerate}
  We get a pre-variety $P$, and the morphisms
  \begin{center}
    \begin{tikzcd}
      & f_1^{-1}(U_2) \cap f_2^{-1}(V_j) \subseteq Z \ar[ddl, bend right=30, swap] \ar[ddr, bend left=30] \ar[d, "f_{i, j}", dashed, swap] \\
      & P \ar[dl] \ar[dr, swap] \\
      X & & Y
    \end{tikzcd}
  \end{center}
  glue to give a morphism
  \begin{center}
    \begin{tikzcd}
      & Z \ar[ddl, bend right=30, swap] \ar[ddr, bend left=30] \ar[d, "f", dashed, swap] \\
      & P \ar[dl] \ar[dr, swap] \\
      X & & Y
    \end{tikzcd}
  \end{center}
  which is a morphism as the morphism
  condition can be checked locally.
  Furthermore, the diagram commutes
  (as can be checked locally). Last,
  $f$ is unique: One can either this check
  locally or check set theoretically
  using $P = \{(x, y) : x \in X \text{ and } y \in Y\}$
  as sets.
\end{proof}

\begin{remark}
  Note that $X \times Y$ is set theoretically
  the product of $X$ and $Y$, but
  not the product of $X$ and $Y$
  as topological spaces.
  Consider $X = Y = \Affine^1$
  and $X \times Y = \Affine^2$.
\end{remark}

\section{Varieties}

\begin{remark}
  We want a version of Hausdorffness in
  algebraic geometry. However, an
  irreducible topological space
  (e.g. $\Affine^n$) is almost never
  Hausdorff (unless it is a single point).
  From a different perspective, note
  that $X$ is Hausdorff if and only if
  the diagonal
  $\Delta_X = \{(x, x) : x \in X\} \subseteq X \times X$ is
  closed, where $X \times X$ is given
  the product topology.
\end{remark}

\begin{definition}
  A pre-variety is \emph{separated} if
  the diagonal
  \[
    \Delta_X = \{(x, x) : x \in X\}
  \]
  is closed in $X \times X$ (the product
  pre-variety). A \emph{variety} is a
  pre-variety that is separated.
\end{definition}

\begin{example}
  $\Affine^n$ is separated. We have
  \[
    V(x_1 - y_1, \dots, x_n - y_n)
    = \Delta_{\Affine^n} \subseteq
    \Affine_{x_i}^n \times \Affine_{y_i}^n
    \cong \Affine^{2n},
  \]
  so $\Delta_{\Affine^n}$ is closed
  in $\Affine^n \times \Affine^n$.
\end{example}

\begin{example}
  Any affine variety is separated. To
  see this, we may assume $X = V(I) \subseteq \Affine^n$.
  By the construction of the product,
  $X \times X \subseteq \Affine^n \times \Affine^n$
  is closed and
  \[
    \Delta_X
    = (X \times X) \cap \Delta_{\Affine^n}.
  \]
  Since $\Delta_{\Affine^n}$ is closed,
  we have
  $\Delta_X$ is closed in $X \times X$ since
  $X \times X$ has the subspace topology.
\end{example}

\begin{prop}
  If $X$ is a variety, then any
  closed or open set $Z \subseteq X$
  is a variety.
\end{prop}

\begin{proof}
  We have already seen that $Z$ is a
  pre-variety, so it suffices to show that
  $Z$ is separated. We note that
  $Z \times Z \hookrightarrow X \times X$
  is an embedding of topological spaces,
  and $\Delta_Z = (Z \times Z) \cap \Delta_X$.
  Since $\Delta_X$ is closed and
  $Z \times Z$ has the subspace topology,
  $\Delta_Z$ is closed in $Z \times Z$.
  So $Z$ is separated.
\end{proof}

\begin{example}
  Recall the bug-eyed line from
  Example \ref{ex:projective-bug-eyed-line}.
  Let $a, b$ be the two origins, and
  write
  $X = U_1 \cup U_2$, where
  $U_1 = X \setminus \{b\} \cong \Affine^1$
  and $U_2 = X \setminus \{a\} \cong \Affine^1$.
  Then consider
  \[
    \Affine^2 \cong U_1 \times U_2 \subseteq X \times X.
  \]
  Note that $\Delta_X \cap (U_1 \times U_2) = \{(x, x) : x \in k \setminus \{0\}\} = \Delta_{\Affine^1} \setminus \{0\}$.
  So $\Delta_X$ is not closed in $X \times X$.
\end{example}

\begin{exercise}
  Show that $\PP_k^1$ is separated.
\end{exercise}

\begin{prop}
  Let $f, g : X \to Y$ be morphisms
  of pre-varieties with $Y$ a variety.
  \begin{enumerate}
    \item The graph $\Gamma_f := \{(x, f(x)) : x \in X\}$
      of $f$
      is closed in $X \times Y$.
    \item $\{x \in X : f(x) = g(x)\}$
      is closed in $X$. This becomes a
      version of the identity principle
      in the case that $X$ is irreducible:
      If $X$ is irreducible and
      $f, g$ agree on a nonempty open set,
      then $f = g$.
  \end{enumerate}
\end{prop}

\begin{proof}
  (1) We can write
  $\Gamma_f = (f, \id)^{-1}(\Delta_Y)$
  where $(f, \id) : X \times Y \to Y \times Y$,
  and $\Delta_Y$ is closed.

  (2) Consider the morphism
  \begin{align*}
    X &\overset{(f, g)}{\longrightarrow} Y \times Y \\
    x &\longmapsto (f(x), g(x)).
  \end{align*}
  Then $\{x \in X : f(x) = g(x)\} = (f, g)^{-1}(\Delta_Y)$,
  so it is closed.
\end{proof}
