\chapter{Aug.~21 --- Hilbert's Nullstellensatz}

\section{Applications of Hilbert's Nullstellensatz}

\begin{corollary}[Weak nullstellensatz]
  If $J \le k[x_1, \dots, x_n]$ is an
  ideal with $J \ne (1)$, then
  $V(J) \ne \varnothing$. Equivalently,
  if $f_1, \dots, f_r \in k[x_1, \dots, x_n]$
  have no common zeros, then there
  exist $g_1, \dots, g_r \in k[x_1, \dots, x_n]$
  such that $\sum_{i = 1}^r f_i g_i = 1$.
\end{corollary}

\begin{proof}
  Assume otherwise that $V(J) = \varnothing$.
  Then
  $I(V(J)) = I(\varnothing) = (1)$,
  so by Hilbert's nullstellensatz,
  we have $\sqrt{J} = (1)$. Then
  $1^n \in J$ for some $n > 0$, so
  $1 \in J$, i.e. $J = (1)$.
\end{proof}

\begin{remark}
  We need
  $k$ to be algebraically closed.
  Note that $(1) \ne (x^2 + 1) \le \R[x]$
  but $V(x^2 + 1) = \varnothing$.
\end{remark}

\begin{corollary}
  There is an inclusion-reversing bijection
  between radical ideals $J \le k[x_1, \dots, x_n]$ and
  affine varieties $X \subseteq \Affine_k^n$
  given by $J \mapsto V(J)$ and $X \mapsto I(X)$.
\end{corollary}

\begin{proof}
  It suffices to show that these maps
  are inverses. For $J \le k[x_1, \dots, x_n]$
  a radical ideal, we have
  \[
    I(V(J)) = \sqrt{J} = J
  \]
  by Hilbert's nullstellensatz. For
  $X \subseteq \Affine_k^n$ an affine
  variety, we clearly have
  $X \subseteq V(I(X)) X$. For the
  reverse inclusion, choose
  an ideal $J \le k[x_1, \dots, x_n]$
  such that $V(J) = X$. Then
  $J \subseteq I(X)$, so we have
  $V(I(X)) \subseteq V(J) = X$.
  Thus we also get $V(I(X)) = X$.
\end{proof}

\begin{remark}
  This implies that maximal ideals in
  $k[x_1, \dots, x_n]$ correspond to
  points in $\Affine_k^n$, since
  maximal ideals correspond to minimal
  varieties under this bijection.
\end{remark}

\begin{corollary}
  If $X_1, X_2$ are affine varieties
  in $\Affine_k^n$, then
  \begin{enumerate}
    \item $I(X_1 \cup X_2) = I(X_1) \cap I(X_2)$;
    \item $I(X_1 \cap X_2) = \sqrt{I(X_1) + I(X_2)}$.
  \end{enumerate}
\end{corollary}

\begin{proof}
  (1) This follows from definitions.

  (2) Write $I(X_1 \cap X_2) = I(V(I(X_1)) \cap V(I(X_2))) = I(V(I(X_1) + I(X_2))) = \sqrt{I(X_1) + I(X_2)}$.
\end{proof}

\begin{example}
  The radical in (2) is necessary. Consider
  $X_1 = V(y)$ and $X_2 = V(y - x^2)$
  in $\Affine_k^2$. Then
  $X_1 \cap X_2 = \{(0, 0)\} \subseteq \Affine_k^2$,
  so $I(X_1 \cap X_2) = (x, y)$.
  However, $I(X_1) + I(X_2) = (y) + (y - x^2) = (y, x^2)$.

  Note that it is sometimes better to
  consider $(y, x^2)$ anyway as
  it tracks multiplicities.
  In particular, we can see the multiplicity
  in the dimension of
  $k[x, y] / (x, y^2) \cong \overline{1} k \oplus \overline{y} k$
  as a $k$-vector space.
\end{example}

\section{Proof of Hilbert's Nullstellensatz}

We will assume the following result
from commutative algebra without proof:
\begin{theorem}[Noether normalization]
  Let $A$ be a finitely generated algebra
  over a field $k$ with $A$ a domain.
  Then there is an injective $k$-algebra
  homomorphism
  $k[z_1, \dots, z_n] \hookrightarrow A$
  that is finite, i.e. $A$ is a
  finitely generated
  $k[z_1, \dots, z_n]$-module.
\end{theorem}

\begin{corollary}
  \label{cor:finite-extension}
  If $K \subseteq L$ is a field extension
  and $L$ is a finitely generated $K$-algebra,
  then $K \subseteq L$ is a finite field
  extension. In particular,
  if in addition $K = \overline{K}$, then
  $K = L$.
\end{corollary}

\begin{proof}
  By Noether normalization,
  there exists a $k$-algebra homomorphism
  $K[z_1, \dots, z_n] \to L$
  that is finite. Then by a result from
  commutative algebra, $L$ is
  integral over $K[z_1, \dots, z_n]$,
  which implies that $K[z_1, \dots, z_n]$
  must also be a field since $L$ is.
  Thus $n = 0$, so $K \subseteq L$ is
  a finite extension.
\end{proof}

\begin{prop}\label{prop:in-maximal-ideal}
  If $(1) \ne J \le R$ is an ideal, then
  $J$ is contained in some maximal ideal.
\end{prop}

\begin{proof}
  Consider the set $P = \{I \le R : J \subseteq I, I \ne (1)\}$
  with the partial order given by inclusion.
  Note that $P \ne \varnothing$ since
  $J \in P$. Furthermore,
  every chain in $P$ has an upper bound
  (for $\{I_\alpha : \alpha \in A\}$ a
  chain $P$, we can take $\bigcup_{\alpha \in A} I_\alpha$, which one can
  check is indeed
  an ideal that lies in $P$; note
  that $1 \notin I_\alpha$ implies
  $1 \notin \bigcup_{\alpha \in A} I_\alpha$).
  So Zorn's lemma implies there is a
  maximal element in $P$, which is a
  maximal ideal.
\end{proof}

\begin{proof}[Proof of Theorem \ref{thm:hilbert-nullstellensatz}]
  We will proceed in the following
  steps:
  \begin{enumerate}
    \item Show that the maximal ideals
      of $k[x_1, \dots, x_n]$ are
      of the form $(x_1 - a_1, \dots, x_n - a_n)$
      for $a_i \in k$.
    \item Prove the weak nullstellensatz:
      If $1 \ne J \le k[x_1, \dots, x_n]$,
      is an ideal, then $V(J) \ne \varnothing$.
    \item Prove the (strong) nullstellensatz:
      $I(V(J)) = \sqrt{J}$
      for $J \le k[x_1, \dots, x_n]$.
  \end{enumerate}
  The most difficult part is the first step
  and is where we need $k$ to be
  algebraically closed.\footnote{The statement is false when $k$ is not algebraically closed: $(x^2 + 1)$ is maximal in $\R[x]$.}

  (1) For $a_1, \dots, a_n \in k$, the
  ideal $(x_1 - a_1, \dots, x_n - a_n)$
  is maximal (the quotient is $k$, which
  is a field). Conversely, fix a maximal
  ideal $\m \in k[x_1, \dots, x_n]$.
  Since
  \[
    k \overset{\phi}{\longrightarrow}
    k[x_1, \dots, x_n] / \m = L
  \]
  is a finitely generated $k$-algebra and
  $k$ is algebraically closed,
  $\phi$ is an isomorphism by
  Corollary \ref{cor:finite-extension}.
  Choose $a_i \in k$ such that
  $\phi(a_i) = \overline{x}_i$, so
  $\overline{x_i - a_i} = 0$ in $L$
  Then $(x_1 - a_1, \dots, x_n - a_n) \subseteq \m$, so they
  must be equal since both the left and
  right hand sides are maximal ideals.

  (2) By Proposition \ref{prop:in-maximal-ideal},
  $J$ is contained in some maximal ideal
  $\m$. By (1), $\m = (x_1 - a_1, \dots, x_n - a_n)$
  for some $a_1, \dots, a_n \in k$.
  Since $J \subseteq \m$, we have
  $V(J) \supseteq V(\m) \supseteq \{(a_1, \dots, a_n)\}$, so
  $J \ne \varnothing$.

  $(3)$ The reverse inclusion follows
  from definitions. For the forward
  inclusion, fix $f \in I(V(J))$, and
  we want to show that $f^n \in J$ for
  some $n > 0$. Add a new variable $y$
  and consider
  \[
    J_1 = (J, fy - 1) \le k[x_1, \dots, x_n, y].
  \]
  Now $V(J_1) = \{(a, b) = (a_1, \dots, a_n, b) \in \Affine_k^{n + 1} : a \in V(J), f(a) b = 1\} = \varnothing$
  since $f$ vanishes on $V(J)$, so
  $f(a) b = 0$ for any $b$. Thus by the
  weak nullstellensatz, $J_1 = (1)$, so
  $1 = \sum_{i = 1}^r g_i f_i + g_0(fy - 1)$
  with $f_1, \dots, f_r \in J$
  and $g_0, \dots, g_r \in k[x_1, \dots, x_n, y]$. Let $N$ be the maximal power
  of $y$ in the $g_i$. Multiplying by
  $f^N$, we get
  \[
    f^N = \sum_{i = 1}^r G_i(x_1, \dots, x_n, fy) f_i + G_0(x_1, \dots, x_n, fy)(fy - 1)
  \]
  with $G_i \in k[x_1, \dots, x_n, fy]$.
  So if we set $fy = 1$, then we have
  \[
    f^N = \sum_{i = 1}^r G_i(x_1, \dots, x_n, 1) f_i + 0 \in J,
  \]
  which gives $f \in \sqrt{J}$. To
  justify this substitution, we can
  consider the quotient
  $k[x_1, \dots, x_n, y] / (fy - 1)$.
  We have a map
  $k[x_1, \dots, x_n] \to k[x_1, \dots, x_n, y] / (fy - 1)$,
  which is injective since $(fy - 1)$
  does not lie in $k[x_1, \dots, x_n]$,
  so an equality in the quotient implies
  an equality in $k[x_1, \dots, x_n]$.
\end{proof}
