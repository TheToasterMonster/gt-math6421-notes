\chapter{Sept.~25 --- Projective Varieties}

\section{Projective Space}

\begin{definition}
  Define \emph{projective $n$-space} over
  $k$ to be
  \[
    \PP^n_k = \PP^n
    = \text{1-dimensional subspaces of $k^{n + 1}$}
    = (k^{n + 1} \setminus \{0\}) / {\sim},
  \]
  where $(x_0, x_1, \dots, x_n) \sim (y_0, y_1, \dots, y_n)$
  if there exists $\lambda \in k^*$
  such that
  $(x_0, \dots, x_n) = \lambda (y_0, \dots, y_n)$.
  We write $[x_0 : x_1 : \dots : x_n] \in \PP^n_k$
  for the equivalence class of
  $(x_0, x_1, \dots, x_n)$.
\end{definition}

\begin{example}
  For $n = 2$, we have
  $[1 : 0 : 2] = [1 / 2 : 0 : 1] \in \PP^2_k$
  when $\Char k \ne 2$.
\end{example}

\begin{remark}
  For $0 \le i \le n$, define
  $U_i = \{[x_0 : x_1 : \dots : x_n] \in \PP^n_k : x_i \ne 0\}$.
  Then
  \[
    \PP^n_k = \bigcup_{i = 0}^n U_i,
  \]
  and there exist bijective maps
  $f_i : U_i \to \Affine^n$ given by
  \[
    f_i([x_0 : \dots : x_n])
    = (x_0 / x_i, \dots, \widehat{x_i / x_i}, \dots, x_n / x_i),
  \]
  where $\widehat{x_i / x_i}$ means we omit
  $x_i / x_i$.
  For $i = 0$, the inverse is
  $f_0^{-1}(x_1, \dots, x_n) = [1 : x_1 : \dots : x_n]$.
\end{remark}

\begin{remark}
  Another way to think about $\PP^n$ is
  via points at $\infty$. Observe that
  \[
    \PP^n \setminus U_0
    = \{[0 : x_1 : \dots : x_n] \in \PP^n_k : (x_1, \dots, x_n) \in k^n \setminus \{0\}\}
    \cong \PP^{n - 1}.
  \]
  So $\PP^n = \Affine^n \sqcup \PP^{n - 1} = \Affine^n \sqcup \Affine^{n - 1} \cup \PP^{n - 2} = \dots = \Affine^n \sqcup \Affine^{n - 1} \sqcup \dots \sqcup \Affine^0$.
\end{remark}

\begin{remark}
  Why work with $\PP^n$? One motivation
  is analytic (e.g. for $k = \C$):
  \begin{enumerate}
    \item $\PP^n_\C$ is compact
      with the analytic topology:
      There are surjective continuous maps
      \begin{center}
        \begin{tikzcd}
          {\R^{2n + 2} \setminus \{0\}} \ar[r, two heads] & {\C \PP^n} \\
          {S^{2n + 1}} \ar[u, hook, swap] \ar[ru, two heads]
        \end{tikzcd}
      \end{center}
    \item \emph{Chow's theorem}:
      Any closed complex submanifold
      of $\C \PP^n$ is a projective
      variety.
  \end{enumerate}
  Another motivation is the extra data
  at $\infty$:
  \begin{enumerate}
    \item If $\ell_1, \ell_2$ are
      distinct lines in $\Affine^2$, then
      $\#(\ell_1 \cap \ell_2) = 0$ or $1$.
      However, over $\PP^2$,
      $\#(\ell_1 \cap \ell_2) = 1$ always.
    \item \emph{Bezout's theorem}: If
      $C_1, C_2 \subseteq \Affine^2$ are
      two distinct irreducible curves in
      $\Affine^2$, then
      \[
        \#(C_1 \cap C_2)
        \le (\deg C_1)(\deg C_2),
      \]
      counting multiplicities. The
      version over $\PP^2$ always gives
      equality.
  \end{enumerate}
\end{remark}

\section{Graded Rings}

\begin{remark}
  In projective space,
  for $f \in k[x_0, \dots, x_n]$,
  we could try to define
  \[
    V(f) = \{[a_0 : \dots : a_n] : f(a_0, \dots, a_n) = 0\}.
  \]
  But this is bad notation as it is
  not well-defined ($f = 0$ depends on
  the representative in the equivalence
  class). Instead, if $f$ is homogeneous
  of degree $d$, then
  \[
    f(\lambda a_0, \dots, \lambda a_n)
    = \lambda^d f(a_0, \dots, a_n),
  \]
  so $V(f)$ is well-defined in this case,
  when $f$ is homogeneous.
\end{remark}

\begin{definition}
  An \emph{$\N$-graded ring} is a ring
  $R$ with subgroups
  $R_d \subseteq R$ for $d \in \N$ such that
  \[
    R = \bigoplus_{d \in \N} R_d
    \quad\text{and}\quad
    R_d R_e \subseteq R_{d + e}.
  \]
  An element $f \in R$ is
  \emph{homogeneous} if there exists
  $d$ such that $f \in R_d$.
\end{definition}

\begin{example}
  For $S = k[x_0, \dots, x_n]$,
  we can take $S_d = \bigoplus_{a_i \ge 0, \sum a_i = d} k x_0^{a_0} \cdots x_n^{a_n}$.
\end{example}

\begin{definition}
  An ideal $I$ in a graded ring is \emph{homogeneous}
  if it is generated by homogeneous elements.
\end{definition}

\begin{example}
  We can write $(x, y^3 - 3x^2) \subseteq k[x, y]$
  as $(x, y^3)$, so it is homogeneous.
\end{example}

\begin{prop}
  Let $R$ be a graded ring with ideal
  $I$. The following are equivalent:
  \begin{enumerate}
    \item $I$ is homogeneous;
    \item for any $f = \sum_{d \in \N} f_d \in I$
      with $f_d \in R_d$, then
      $f_d \in I$ for all $d$;
    \item $I = \bigoplus_{d \in \N} (I \cap R_d)$.
  \end{enumerate}
\end{prop}

\begin{proof}
  Left as an exercise. The interesting
  implication is $(1 \Rightarrow 2)$.
\end{proof}

\begin{prop}
  Let $I, J$ be homogeneous ideals
  of a graded ring $R$. Then
  \begin{enumerate}
    \item $I + J$, $IJ$, $\sqrt{I}$, and
      $I \cap J$ are all homogeneous;
    \item $R / I$ is a graded ring
      with $R / I = \bigoplus_{d \in \N} R_d / I_d$, where
      $I_d = I \cap R_d$.
  \end{enumerate}
\end{prop}

\begin{proof}
  (1) We prove that $\sqrt{I}$ is
  homogeneous.
  Assume $f \in \sqrt{I}$, and write
  $f = f_0 + f_1 + \dots + f_d$ with
  $f_i \in R_i$ and $f_d \ne 0$. Now
  there exists $n > 0$ such that
  $f^n \in I$, and
  \[
    f^n = f^n_d + \text{lower order terms}.
  \]
  Since $I$ is homogeneous, $f_d^n \in I$,
  so $f_d \in \sqrt{I}$.
  Then $f_0 + \dots + f_{d - 1} \in \sqrt{I}$,
  and we can repeat.

  (2) We can write
  $R / I = (\bigoplus_{d \in \N} R_d) / (\bigoplus_{d \in \N} (I \cap R_d))$.
  As abelian groups, this is
  $R / I \cong \bigoplus_{d \in \N} R_d / I_d$.
  One can check that the multiplication
  also respects the grading, so
  this is an isomorphism of rings.
\end{proof}

\section{Projective Varieties}

\begin{definition}
  For a set $T \subseteq k[x_0, \dots, x_n]$ of
  homogeneous elements, define
  its \emph{vanishing locus}
  \[
    V_p(T) := V(T)
    = \{[x_0 : \dots : x_n] \in \PP^n : f(x) = 0 \text{ for all } f \in T\}
    \subseteq \PP^n.
  \]
  A \emph{projective variety}
  is a subset of this form.
  For a homogeneous ideal $I \le k[x_0, \dots, x_n]$,
  define
  \[
    V(I) = \{x \in \PP^n : f(x) \text{ for all } f \in I \text{ homogeneous}\}.
  \]
  For a subset $X \subseteq \PP^n$, define
  its \emph{ideal}
  \[
    I_p(X) := I(X)
    = ( f \in k[x_0, \dots, x_n] \text{ homogeneous} : f(x) = 0 \text{ for all } [x] \in X ).
  \]
\end{definition}

\begin{remark}
  If $T \subseteq k[x_0, \dots, x_n]$
  is a subset of homogeneous elements,
  then we have $V_p(T) = V_p((T))$.
  So projective varieties can
  equivalently be defined as vanishing
  sets of homogeneous ideals.
\end{remark}

\begin{example}
  Consider $X = V_p(x^2 - yz) \subseteq \PP^{2}_{x : y : z}$.
  Set $H = V(x)$, then there is a bijection
  \begin{align*}
    U = \PP^2 \setminus H
    &\overset{f}{\longrightarrow} \Affine^2 \\
    [1 : y : z]
    &\longmapsto (y, z).
  \end{align*}
  Then $f(X \cap U) = V(1 - yz)$. On the
  other hand, we can see that
  \[
    X \cap H
    = \{[0 : 1 : 0], [0 : 0 : 1]\}
    = \{a, b\}.
  \]
  If we were working with $\C$ with
  the analytic topology, then we can
  take limits on $V(1 - yz)$ and see
  \[
    \lim_{t \to 0} [1 : t : 1 / t]
    = \lim_{t \to 0} [t : t^2 : 1]
    = [0 : 0 : 1] = b.
  \]
  Note that we essentially switched
  charts in order to take this limit.
  Similarly, we have
  \[
    \lim_{t \to \infty}
    [1 : t : 1 / t]
    = \lim_{t \to \infty} [1 / t : 1 : 1 / t^2]
    = [0 : 1 : 0] = a.
  \]
  So we can see $a, b$ as points at
  $\infty$ compactifying the curve
  $V(1 - yz)$.
\end{example}

\begin{example}
  We have the following:
  \begin{enumerate}
    \item $V_p(0) = \PP^n$;
    \item $V_p(1) = \varnothing$;
    \item if $p = [a_0 : \dots : a_n]$
      and
      $J = (a_i x_j - a_j x_i : 0 \le i, j \le n)$,
      then $V(J) = \{0\}$;
    \item $I_0 = (x_0, \dots, x_n)$
      is called the \emph{irrelevant ideal},
      which has
      $V_p(I_0) = \varnothing = V_p(1)$
      but $I_0 = \sqrt{I_0} \subsetneq (1)$.
  \end{enumerate}
\end{example}
