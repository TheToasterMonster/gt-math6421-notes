\chapter{Sept.~2 --- Dimension}

\section{More on Dimension}

\begin{remark}
  Recall the following correspondence
  from before:
  If $X \subseteq \Affine_k^n$ is an
  affine variety, then there exists a
  bijection between the
  irreducible closed subsets
  $Y \subseteq X$ and the prime ideals
  $\p \le A(X)$.
\end{remark}

\begin{definition}
  For a ring $A$, the \emph{(Krull) dimension}
  of $A$, denoted $\dim A$, is the
  supremum of the $n$ such that there
  exists a chain of prime ideals
  \[
    A \supseteq \p_0 \supsetneq \p_1 \supsetneq \cdots \supsetneq \p_n.
  \]
  For a prime ideal $\q \le A$, the
  \emph{height} of $\q$, denoted
  $\height \q$, is the supremum of the
  $n$ as above with $\p_0 = \q$.
\end{definition}

\begin{remark}
  If $X$ is an affine variety, then
  we have the following:
  \begin{enumerate}
    \item $\dim X = \dim A(X)$;
    \item for $Y \subseteq X$ a
      closed irreducible subset,
      $\codim_X Y = \height I_X(Y)$.
  \end{enumerate}
  These properties follow from the
  inclusion-reversing correspondence.
\end{remark}

\begin{definition}
  Let $K \subseteq L$ be a field extension.
  \begin{enumerate}
    \item A collection of elements
      $\{z_i : i \in I\} \subseteq L$
      is a \emph{transcendence basis}
      of $K \subseteq L$ if
      the $z_i$ are algebraically
      independent (i.e. $K(x_i : i \in I) \xrightarrow{\cong} K(z_i : i \in I)$ by $x_i \mapsto z_i$)
      and $K(z_i : i \in I) \subseteq L$
      is algebraic.
    \item The \emph{transcendence degree}
      $\trdeg_K L$ is the cardinality
      of a transcendence basis.
  \end{enumerate}
\end{definition}

\begin{theorem}[Dimension theory]
  Let $A$ be a finitely generated
  $k$-algebra that is a domain.
  Then
  \begin{enumerate}
    \item $\dim A = \trdeg_k \Frac(A)$;
    \item for any prime ideal
      $\p \le A$, we have
      $\height \p + \dim A / \p = \dim A$;
    \item all maximal chains of
      prime ideals
      $A \supseteq \p_0 \supsetneq \p_1 \supsetneq \cdots \supsetneq \p_n$
      are of the same length.
  \end{enumerate}
\end{theorem}

\begin{remark}
  The following are consequences of the
  above result from commutative algebra:
  \begin{enumerate}
    \item $\dim_k \Affine_k^n = \dim k[x_1, \dots, x_n] = \trdeg_k k(x_1, \dots, x_n) = n$.
    \item If $X$ is irreducible, then
      $A(X)$ is a domain, so for
      $x \in X$, we have
      \[
        \codim_X \{x\}
        = \height I(\{x\})
        = \dim A(X) - \dim A(X) / I(\{x\})
        = \dim A(X) = \dim X,
      \]
      where we note that $A(X) / I(\{x\}) \cong k$
      is a field.
    \item If $X$ is an irreducible affine
      variety and $U \subseteq X$ is a
      nonempty open subset, then
      \[
        \dim U = \sup_{x \in U} \codim_U \{x\}
        = \sup_{x \in U} \codim_X \{x\}
        = \dim X.
      \]
      This follows since we can
      pass from a chain in $U$ to a
      chain in $X$ by taking closures.
    \item If $X$ is an irreducible
      affine variety and $Z \subseteq X$
      is an irreducible closed subset, then
      \[
        \dim Z = \dim X - \codim_X Z.
      \]
  \end{enumerate}
  Note that (2)-(4) can be false if
  $X$ is not irreducible. To contradict
  (4),
  let $X = V(x, y) \cup V(z) \subseteq \Affine_k^3$ with
  $Z = V(x, y)$. Then we have
  $\dim X = 2$, $\dim Z = 1$, $\codim_X Z = 0$.
\end{remark}

\section{Hypersurfaces}

\begin{remark}
  We now want to study hypersurfaces.
\end{remark}

\begin{theorem}[Krull's Hauptidealsatz]
  If $A$ is a Noetherian ring and
  $f \in A$ is nonzero and a non-unit,
  then every minimal prime ideal containing
  $f$ has height $1$.
\end{theorem}

\begin{corollary}\label{cor:hypersurface-dim}
  If $X \subseteq \Affine_k^n$ is an
  irreducible affine variety and
  $f \in A(X)$ is a nonzero non-unit, then
  \[
    \dim Z = \dim X - 1
  \]
  for every irreducible component
  $Z$ of $V_X(f)$.
\end{corollary}

\begin{proof}
  Since $X$ is irreducible, $A(X)$ is a
  domain. So there is a correspondence
  between the minimal prime ideals
  $f \in \p \subsetneq A(X)$ and the
  minimal irreducible closed subsets
  $Z \supseteq V_X(f)$, which corresponds
  to the irreducible components
  $Z$ of $V_X(f)$. For such a component
  $Z$, we know
  \[
    \dim Z = \dim Z - \codim_X Z
    = \dim X - \height I(Z)
    = \dim X - 1
  \]
  by Krull's Hauptidealsatz, which is
  the desired result.
\end{proof}

\begin{example}
  Corollary \ref{cor:hypersurface-dim} implies that if
  $f \in k[x_1, \dots, x_n]$ is
  non-constant, then
  \[
    \dim V(f) = \dim \Affine_k^n - 1
    = n - 1.
  \]
\end{example}

\begin{theorem}
  An irreducible affine variety
  $Y \subseteq \Affine_k^n$ has
  $\dim Y = n - 1$ if and only if
  $Y = V(f)$ for some non-constant
  polynomial $f \in k[x_1, \dots, x_n]$.
\end{theorem}

\begin{proof}
  $(\Leftarrow)$ This
  was Corollary \ref{cor:hypersurface-dim}.

  $(\Rightarrow)$ We will use that
  $A(\Affine_k^n) = k[x_1, \dots, x_n]$
  is a UFD. Since $Y$ is irreducible
  and $\dim Y = n - 1$,
  \[
    \height I(Y)
    = \codim_{\Affine_k^n} Y
    = \dim \Affine_k^n - \dim Y
    = 1.
  \]
  Since $(0) \subsetneq I(Y) \subsetneq k[x_1, \dots, x_n]$,
  there exists a non-constant
  $f \in k[x_1, \dots, x_n]$ with
  $f \in I(Y)$. Write
  \[
    f = f_1 \cdots f_r
  \]
  with $f_i$ irreducible by
  unique factorization, and note that
  the $f_i$ are also prime since we are in
  a UFD. Since $I(Y)$ is prime,
  some $f_i$ is in $I(Y)$, so we have
  the inclusions
  \[
    (0) \subsetneq (f_i) \subseteq I(Y).
  \]
  Since $\height I(Y) = 1$, we must have
  $(f_i) = I(Y)$, so
  $Y = V(I(Y)) = V(f_i)$.
\end{proof}

\section{Regular Functions}

\begin{definition}
  Let $X$ be an affine variety and
  $U \subseteq X$ open. A function
  $\varphi : U \to k$ is \emph{regular} if
  for each $a \in U$, there exists an
  open neighborhood $a \in U_a \subseteq U$
  and $f, g \in A(X)$ such that
  \[
    \varphi(x) = \frac{g(x)}{f(x)},
    \quad f(x) \ne 0,
    \quad \text{for all } x \in U_a.
  \]
  Define $\OO_X(U) = \{\varphi : U \to k \mid \varphi \text{ is a regular function on } U\}$.
\end{definition}

\begin{exercise}
  Check that $\OO_X(U)$ is a
  ring under pointwise addition and
  multiplication of outputs.
\end{exercise}

\begin{remark}
  To patch open sets together, we will
  later need the notion of a \emph{morphism},
  and
  a morphism $U \to Y \subseteq \Affine_k^m$
  should be given by
  \[
    x \longmapsto (\varphi_1(x), \dots, \varphi_m(x))
  \]
  with $\varphi_i$ regular functions on $U$.
\end{remark}

\begin{example}
  We have the following:
  \begin{enumerate}
    \item If $X \subseteq \Affine_k^n$
      is an affine variety, then any
      $\varphi \in A(X)$ is regular.
      Furthermore, we get an injective ring homomorphism
      $A(X) \to \OO_X(X)$. We will see that
      this is an isomorphism.
    \item If $X = \Affine_k^1$ and
      $U = \Affine_k^1 \setminus \{0\}$,
      then for any $n \ge 0$ and
      $g \in k[x]$, the function
      $g / x^n$ is regular on $U$.
    \item Let $X = V(x_1x_4 - x_2 x_3) \subseteq \Affine_k^4$ and
      $U = X \setminus V(x_2, x_4)$.
      Then the following map is regular:
      \begin{align*}
        \varphi : U &\longrightarrow k \\
        (x_1, x_2, x_3, x_4) &\longmapsto
        \begin{cases}
          x_1 / x_2, & \text{if } x_2 \ne 0, \\
          x_3 / x_4, & \text{if } x_4 \ne 0.
        \end{cases}
      \end{align*}
      Note that on $U \setminus V(x_2 x_4)$,
      we have $x_1 / x_2 = x_3 / x_4$ since
      $x_1 x_4 = x_2 x_3$ on $X$.
  \end{enumerate}
\end{example}
